

\documentclass{article}

\usepackage[utf8]{inputenc}
\title {}
\author{Pérez Villarroya, David }

\begin{document}

\maketitle{}


\section {Objetivos}

La actual capacidad de generar gran cantidad de información genómica, producida por el gran desarrollo de las tecnologías de secuenciación de ADN, también implican un necesidad de nuevas herramientas computacionales para analizar e interpretar estos datos dentro del contexto ecológico y evolutivo, con el fin de obtener la visón más integrativa posible sobre la triple interacción genotipo, fenotipo y ambiente. Dentro de este contexto, se situa la secuencición de ADN ambiental o metagenómica que, apoyada sobre el desarrollo de las tecnologías de secuenciación de ADN, ha permitido superar la dependencia de cultivo para la caracterización genómica microbiana, así como, mostrar la gran diversidad microbiana existente en la fracción de especies microbianas no cultivables.

Sin embargo, a pesar de la facilidad para la obtención de datos metagenómicos, la limitación se encuentra en el análisis de los mismos. Uno de los principales retos de este análisis, es determinar que especies y en que proporción se encuentran dentro de las secuencias obtenidas, ya que las muestras metagenómicas abarcan diversos conjuntos de poblaciones microbianas presentes en el ambiente, lo cual, dificulta el ensamblado de las secuencias, a lo que se suma, el factor limitante de longitud de las secuencias obtenidas. Se hace necesario, por tanto, un avance en las técnicas de análisis computacional de datos metagenómicos para superar esta limitación en la asignación taxonómica de las secuencias y mejorar con ello el posterior ensamblado y análisis de la información contenida en las secuencias metagenómicas.







\end{document}
